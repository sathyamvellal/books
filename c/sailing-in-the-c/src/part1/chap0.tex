\chapter{Hello World}

%%% Agenda %%%
% A Hello World program
% Dissection of the hello world program
% Hence, how are C programs structured on the most fundamental level

\section{Agenda}
\begin{itemize}
 \item Hello World Program.
 \item Dissection of the hello world program.
 \item Fundamental structure of a C program.
\end{itemize}

\section{Program \#1: Hello World}
And finally, after all that theory, we're now down to business. \textbf{Hello World} is a tradition I must embrace! So I'm now going to present to you a minimalistic program in C which simply displays "Hello World". What I'll do, is show you the program and then explain what it does. So here it goes - \\

\begin{ccode}
#include <stdio.h>

int main()
{
  printf ("Hello World\n");
  return 0;
}
\end{ccode}
\\

Okay, so that's like 7 lines of code but it says a lot about a C Program! So let's start the dissection.

\begin{itemize}
 \item We begin the program with the line \hilight{\#include <stdio.h>}. What this does is include a "library header file". I'll talk about header files soon. Just know for now that this is required to use the function \hilight{printf} which we use to display stuff on the console. 
 \item We are then declaring the \hilight{main} function. 
 \item We use \hilight{printf} to display "Hello World".
 \item \hilight{return 0} is to indicate that our program executed succesfully without any errors.
\end{itemize}
