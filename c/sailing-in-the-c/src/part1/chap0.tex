\chapter{Hello World}

%%% Agenda %%%
% A Hello World program
% Dissection of the hello world program
% Hence, how are C programs structured on the most fundamental level

\section{Agenda}
\begin{itemize}
 \item Hello World Program.
 \item Dissection of the hello world program.
 \item Fundamental structure of a C program.
\end{itemize}

\section{Program \#1: Hello World}
And finally, after all that theory, we're now down to business. \textbf{Hello World} is a tradition I must embrace! So I'm now going to present to you a minimalistic program in C which simply displays ``Hello World". What I'll do, is show you the program and then explain what it does. So here it goes - \\

\begin{ccode}
#include <stdio.h>

int main()
{
  printf ("Hello World\n");
  return 0;
}
\end{ccode}
\\

Okay, so that's like 7 lines of code but it says a lot about a C Program! Let's look at each aspect in brief and then in detail. By the way, I hope you remember the 2 sets of points which I discussed in the previous chapter, you can see them in action in here.

\begin{itemize}
\item Firstly we include a file called \textit{stdio.h} which is a header file. This is done by the \textit{\#include} preprocessor directive.
\item We then define the \textit{main} function. Hope you remember from the previous chapter that this function is the entry to our program. 
\item In \textit{main} we have two \textbf{statements} which are analogous to meaningful sentences in English, for a programming language. 
\item The first statement prints a \textit{string} - "Hello World\n" on the screen. 
\item The second statement returns an \textit{integer} back (back to whom? We'll see that in a short while), the integer in this case being zero. 
\end{itemize}

\begin{itemize}
\item 
\end{itemize}
