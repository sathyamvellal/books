\chapter{Hello World}

%%% Agenda %%%
% A Hello World program
% Dissection of the hello world program
% Hence, how are C programs structured on the most fundamental level

\section{Agenda}
\begin{itemize}
 \item Hello World Program.
 \item Dissection of the hello world program.
 \item Fundamental structure of a C program.
\end{itemize}

\section{Program \#1: Hello World}
And finally, after all that theory, we're now down to business. \textbf{Hello World} is a tradition I must embrace! So I'm now going to present to you a minimalistic program in C which simply displays ``Hello World". What I'll do, is show you the program and then explain what it does. So here it goes - \\

\begin{ccode}
#include <stdio.h>

int main()
{
  printf ("Hello World\n");
  return 0;
}
\end{ccode}
\\

Okay, so that's like 7 lines of code but it says a lot about a C Program! Let's look at each aspect in brief and then in detail. By the way, I hope you remember the 2 sets of points which I discussed in the previous chapter, you can see them in action in here.

\begin{itemize}
\item Firstly we include a file called \hilight{stdio.h} which is a header file. This is done by the \hilight{\#include} preprocessor directive.
\item We then define the \hilight{main} function. Hope you remember from the previous chapter that this function is the entry to our program. 
\item In \hilight{main} we have two \textbf{statements} which are analogous to meaningful sentences in English, for a programming language. 
\item The first statement has a function being called, \hilight{printf} which prints a \textit{string} - "Hello World\n" on the screen. 
\item The second statement returns an \textit{integer} back (back to whom? We'll see that in a short while), the integer in this case being zero. 
\end{itemize}

That's an overview of the program that has been presented above. So now let's got on to things in detail!

\begin{itemize}
\item All preprocessor directives begin with \hilight{\#}. What \hilight{\#include} does is include the contents of the file mentioned. If the filename is enclosed within angular brackets, then its present in the library path (typically /usr/include). The file that we are including is \hilight{stdio.h} which is a header file for the Standard Input and Output. Header files are like blueprints. To construct a building, architects design a blueprint first. This helps them visualize how the building will be. Note that its not constructed yet. As a customer, I don't know how it'll look, the architect has that picture in mind. All that I know is how the house will be structured, how many rooms it has, what kind of a design its going to be built with, the color of walls and the ceiling, the flooring and everything else. But I do not have the picture. I do not have anything concrete on how the building will look like till its finished. But the architect who designed it knows best. A header file, similarly, contains blueprints (also called prototype) to a set of functions. \hilight{stdio.h} contains such prototypes for all input and output functions in the C language, That means, the function that printed on the screen in this program, \hilight{printf}, has its prototype declared in \higlight{stdio.h}. And why? Because its one of the standard output functions.
\end{itemize}
