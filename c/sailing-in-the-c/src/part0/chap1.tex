\chapter{Preliminary Concepts}

%%% AGENDA %%%
% Programming Languages
% Programming Language Paradigms
% Which paradigm is C in? Why?
% Hence, When to code in C?
% Some general guidelines when programming in any language
% Some general guidelines when programming in C

In this chapter, I'll be talking about some preliminary concepts which I believe will help in understanding both the C language and soem aspects of programming. If you are completely knew to programming, I recommend you to read this chapter.

So let's begin. I consider programming to be an art. And just like art, there is a structured way to learn good programming. Many people want some problem to be solved and all they do is try to learn some programming language and try to implemenet the solution they designed. It is however important to know when implementing a solution, the programming language also plays a very important solution. Are you wondering what this is? There exists thousands of natural languages all over the world. They were designed mainly to communicate. But programming languages are carefully designed to solve a purpose and reach a broac variety of problem sets. So its always very important to keep in mind the reasons behind the choice of the programming language, the way the implementation needs to be to be effective so that it can exploit the features provided by the language. This makes us better programmers. Not just programming in some language. 

You already know why C came into existence from the previous chatper. What we'll do now is explore how C is structured so that when I actually get down to explaining C, you'll be in a better position to understand. For this, we'll first explore how C is structured. 

Important points first
\begin{itemize}
\item Programs are written in a combination of \textbf{source files (.c)} and \textbf{"header files" (.h)}
\itme Just like how we have a main gate for a building, there's a \textbf{main} function which is the entry to every C Program on this planet; unless of course martians have something else!
\item Just like how you have different rooms, each serving their own purpose in a house, a C program has smaller computational entities called \textbf{functions}. The \textit{main}, mentioned in the previous point, is a function.
\item Just like how we have items and objects in the house representing something, we have representational entities which are usually called \textbf{variables}. 
\end{itemize}

If you didn't get the basic overview of a C Program from above, that's fine. You have tons of chapters ahead to take you deep into the C!. 